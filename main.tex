\documentclass{article}
\usepackage[utf8]{inputenc}
\usepackage[spanish]{babel}
\usepackage{listings}
\usepackage{graphicx}
\graphicspath{ {images/} }
\usepackage{cite}

\begin{document}

\begin{titlepage}
    \begin{center}
        \vspace*{1cm}
            
        \Huge
        \textbf{Nociones de la memoria del
computador}
            
        \vspace{0.5cm}
        \LARGE
        Taller De Memoria
            
        \vspace{1.5cm}
            
        \textbf{Jorge Andrés Montaña Cisneros}
            
        \vfill
            
        \vspace{0.8cm}
            
        \Large
        Despartamento de Ingeniería Electrónica y Telecomunicaciones\\
        Universidad de Antioquia\\
        Medellín\\
        Septiembre de 2020
            
    \end{center}
\end{titlepage}

\tableofcontents

\newpage

\section{Memoria Del Computador}
La memoria de un computador es el dispositivo donde se almacena información principalmente datos e instrucciones de manera temporal con la cual trabajan los microprocesadores o CPU para procesarla y posteriormente devolver los resultados de los procesos ejecutados. \cite{augus}

\section{Tipos De Memoria} \label{contenido}

\subsubsection{Memoria RAM}
La memoria RAM (Random Access Memory) es un tipo de memoria que almacena información de manera no secuencial esto quiere decir que los datos pueden ser accedidos en cualquier orden, indistintamente de su posición o dirección; Esto es debido a que la misma está dividida en celdas de memoria donde se almacenan cada uno de los bits o pulsos eléctricos.\cite{ramwebsite}

\subsubsection{Memoria ROM}
la memoria ROM  se caracteriza por ser únicamente de acceso para lectura y nunca para escritura, además de tener un acceso secuencial; una de sus funciones es almacenar todos los parámetros necesarios para el encendido de la PC.\cite{romwebsite}

\subsubsection{Memoria caché}
La memoria Cache se utiliza para trabajar con los datos e instrucciones que el microprocesador ve que se utilizan más seguido, para ello crea una copia de esos datos en la memoria Caché para tenerlos siempre a mano.

\subsubsection{Disco Duro}
El disco duro es un tipo de memoria de almacenamiento de datos no volátil, es decir que la información almacenada no es borrada aunque no, se encuentre energizado; es capaz de almacenar gran cantidad de información aunque su velocidad es demasiado lenta comparada con otros tipos de memoria.\cite{discwebsite} 

\section{Gestion De Memoria Del Computador}

La memoria de un computador es gestionada a través de un controlador ubicado en la placa madre entre los módulos de memoria y la CPU o incorporado directamente dentro del microprocesador, este controlador de memoria se encarga de gestionar e intervenir en cada transferencia de información que entra o sale de la memoria, es un mediador que comunica las instrucciones del microprocesador o CPU. Por otro lado es capaz de establecer la velocidad con la que se estan realizando las operaciones, las cuales generalmente son medidas en Megahertz (Mhz).\cite{augus}
\section{Rapidez De Una Memoria}
las memorias por lo general tienden a varias en algunos aspectos dependiendo del tipo que sean, desde la cantidad de almacenamiento hasta la velocidad, una de las principales razones de porque una memoria es más rápida que otra tiene que ver con la frecuencia y la latencia, la primera se refiere a la cantidad de veces por segundo que se tarda en hacer una operación,  y la segunda  mide la cantidad de tiempo que se tarda en obtener de la memoria cada bit de información, o sea el tiempo que pasa desde que el controlador de memoria pide en nombre del microprocesador una serie de datos y dichos datos son obtenidos.Estos dos aspectos determinan que tan rápido la memoria podra trabajar con los datos pero  siempre se debe buscar un equilibrio entre ambos para que el rendimiento sea el mas óptimo. \cite{augus}

\subsubsection{¿Porque es importante?}
La rapidez de una memoria es muy importante en un PC por diversas razones, la principal es que nos permite una mejor autonomía y rendimiento, permitiendo trabajar con una gran cantidad de datos de información al tiempo, brindando una experiencia mas agradable y fiable.


\newpage

\bibliographystyle{IEEEtran}
\bibliography{references}

\end{document}

